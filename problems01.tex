\documentclass[12pt]{amsart}

\begin{document}

\def\Z{{\mathbb Z}}
\def\Q{{\mathbb Q}}

\centerline{\bf Mathematics 361:  Number Theory}
\centerline{\bf Assignment \#1}

\vskip .3in

{\bf Reading:  }Ireland and Rosen, Chapter 1 (including the exercises)

\vskip .2in

{\bf Problems:}

\vskip .1in

Euclidean algorithm and linear Diophantine equations:

1. Let $0<b<a$.  The Euclidean algorithm is:
\begin{itemize}
\item (Initialize) Set
$$
[x,y;\alpha,\beta,\gamma,\delta;s]=[a,b;1,0,0,1;0].
$$
\item (Divide) We have $x=qy+r$, $0\le r<y$; set
$$
[x,y;\alpha,\beta,\gamma,\delta;s]=
[y,r;\gamma,\delta,\alpha-q\gamma,\beta-q\delta;s+1].
$$
If $y=0$, go to the next step; otherwise repeat this step.
\item (Output) Return $x$; $\alpha$, $\beta$; $s$.  Here
  $x=\gcd(a,b)=\alpha a+\beta b$, and the running time is~$s$.
\end{itemize}

(a) Show that after the initialization step,
$$
(x,y)=(a,b),\quad x=\alpha a+\beta b,\quad y=\gamma a+\delta b.
$$

(b) Show that each division step preserves the conditions by showing
that
\begin{align*}
(x_{\rm new},y_{\rm new})&=(a,b),\\
x_{\rm new}&=\alpha_{\rm new}a+\beta_{\rm new}b,\\
y_{\rm new}&=\gamma_{\rm new}a+\delta_{\rm new}b,
\end{align*}
given that these relations are established with ``old'' instead of~``new''
throughout.

(c) Show that at termination the conditions are
\begin{align*}
(x)&=(a,b),\\
x&=\alpha a+\beta b.
\end{align*}
Thus $x=\gcd(a,b)$ (the positive greatest common divisor), and we have
expressed $\gcd(a,b)$ as a linear combination of $a$ and~$b$.

(d) The algorithm generates a succession of remainders
\begin{align*}
r_{-1}&=a,\\
r_0&=b,\\
r_k&=r_{k-2}-q_kr_{k-1},\quad k=1,\cdots,s,
\end{align*}
with each $q_k\ge1$ and
$$
r_{-1}>r_0>r_1>\cdots>r_{s-1}>r_s=0,\quad s\ge1.
$$
Here $s$ is the number of steps that the algorithm takes.
Let $F_0=0$, $F_1=1$, $F_2=1$, $F_3=2$, and so on be the Fibonacci numbers.
Thus we have
\begin{align*}
r_{s-1}&\ge1=F_2,\\
r_{s-2}&\ge2=F_3,\\
r_{s-3}&\ge r_{s-2}+r_{s-1}\ge F_4,\\
&\ \,\vdots\\
b=r_0=r_{s-s}&\ge F_{s+1}.
\end{align*}
A lemma (see page~72 of Jamie Pommersheim's book) that you may take
for granted or prove says that $F_{k+2}>\varphi^k$ for~$k\ge1$, where
$\varphi$ is the Golden Ratio.
Show that consequently an integer upper bound on the number~$s$ of
steps for the Euclidean algorithm to compute $\gcd(a,b)$ where
$0<b<a$ is
$$
\boxed{\lceil\log_\varphi(b)\rceil\ge s.}
$$

(e) Work Ireland and Rosen, Exercises 1.3, 1.5---1.8, 1.13, 1.14.
For~1.13, let $g$ be the generator of the ideal generated by the~$n_i$
and argue that $g$ is the gcd of the~$n_i$.  Then use this idea
in~1.14.  Also, 1.6 can be done tidily by using ideals.

\bigskip

Some ring theory:

2. (a) Let $R$ be a commutative ring with $1$.  Show that $R$ is an integral 
domain if and only if the cancellation law holds.

(b) Show that if $R$ is a field then $R$ is an integral domain.

3. Prove that $\Q(i)=\Q[i]$ and $\Q(\omega)=\Q[\omega]$.

4. Consider the ring $R=\Z[\sqrt{-5}]$. 
Show that the ideal $(2, 1+\sqrt{-5})$ is not principal, so $R$ is not a PID.
Use the norm $N(x+y\sqrt{-5})=x^2+5y^2$ to show that $2$ is irreducible in $R$
but not prime in $R$ since $2\mid6=(1+\sqrt{-5})(1-\sqrt{-5})$.

\vfil\pagebreak

Mersenne primes and Fermat primes, cf.\ Ireland and Rosen, 
Exercises 1.24---1.26:

5.  Let $a\ge2$ and $n\ge2$.  Use the geometric sum formula and its variant
$$
r^n-1=(r-1)\sum_{j=0}^{n-1}r^j,
\qquad
r^n+1=(r+1)\sum_{j=0}^{n-1}(-1)^jr^j\quad\text{for $n$ odd}
$$
to prove that (a) if $a^n-1$ is prime then $a=2$ and $n$ is prime (such 
$2^p-1$ primes are called {\sl Mersenne primes}); (b) if $a^n+1$ is prime then
$a$ is even and $n$ is a power of $2$ (in particular, $2^{2^n}+1$ primes are 
called {\sl Fermat primes}).

Incidentally, the geometric sum formula and its variant quickly yield the 
identities
$$
x^n-y^n=(x-y)\sum_{j=0}^{n-1}x^{n-1-j}y^j
$$
and
$$
x^n+y^n=(x+y)\sum_{j=0}^{n-1}(-1)^jx^{n-1-j}y^j\quad\text{for $n$ odd},
$$
which should be familiar from high school for small values of $n$.

\vskip .1in

No polynomial generates a sequence of prime values:

6. Let $f$ be a nonconstant polynomial with integer coefficients.  

(a) If $f$ has degree $n$ show that
$$
f(x+h) = f(x) + \frac{f'(x)}{1!}h + \frac{f''(x)}{2!}h^2 + \cdots  + 
\frac{f^{(n)}(x)}{n!}h^n.
$$
(One can show this using Taylor's Theorem with Remainder or prove it as a
formal polynomial identity.)  Note that each $f^{(j)}(x)/j!$ also has integer 
coefficients.

(b) Show that the sequence
$$
\{f(1), f(2), f(3), \dots\}
$$
does not consist solely of primes past any starting index, as follows.  
Without loss of generality, the leading coefficient of $f$ is positive, so 
$f(n_0)>1$ for some integer $n_0$ beyond which $f$ is monotone increasing; 
then $f(n_0+kf(n_0))$ is composite for all $k\ge1$.

(The polynomial expression $x^2-x+41$ is prime for $0\le x\le 40$.)

%\vskip .1in
%
%7. Work some other problems from Ireland and Rosen, Chapter 1.

\end{document}
